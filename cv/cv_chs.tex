%% start of file `template-zh.tex'.
%% Copyright 2006-2012 Xavier Danaux (xdanaux@gmail.com).
%
% This work may be distributed and/or modified under the
% conditions of the LaTeX Project Public License version 1.3c,
% available at http://www.latex-project.org/lppl/.


\documentclass[11pt,a4paper,sans]{moderncv}   % possible options include font size ('10pt', '11pt' and '12pt'), paper size ('a4paper', 'letterpaper', 'a5paper', 'legalpaper', 'executivepaper' and 'landscape') and font family ('sans' and 'roman')

% moderncv 主题
\moderncvstyle{classic}                        % 选项参数是 ‘casual’, ‘classic’, ‘oldstyle’ 和 ’banking’
\moderncvcolor{blue}                          % 选项参数是 ‘blue’ (默认)、‘orange’、‘green’、‘red’、‘purple’ 和 ‘grey’
%\nopagenumbers{}                             % 消除注释以取消自动页码生成功能

% 字符编码
\usepackage[utf8]{inputenc}                   % 替换你正在使用的编码
\usepackage{CJKutf8}

% 调整页面出血
\usepackage[scale=0.8]{geometry}
%\setlength{\hintscolumnwidth}{3cm}           % 如果你希望改变日期栏的宽度

% 个人信息
\firstname{钟巧勇}
\familyname{}
%\firstname{巧勇}
%\familyname{钟}
%\title{简历题目 (可选项)}                      % 可选项、如不需要可删除本行
\address{上海市岳阳路320号}{200031}             % 可选项、如不需要可删除本行
\mobile{+86~150~2132~9454}                         % 可选项、如不需要可删除本行
\phone{+86~21~5492~0235}                          % 可选项、如不需要可删除本行
%\fax{+3~(456)~789~012}                            % 可选项、如不需要可删除本行
\email{solary.sh@gmail.com}                    % 可选项、如不需要可删除本行
\homepage{xiaoyong.org}                  % 可选项、如不需要可删除本行
\extrainfo{1988年01月26日出生于浙江}                  % 可选项、如不需要可删除本行
\photo[56pt][0.4pt]{xiaoyong.jpg}                  % ‘64pt’是图片必须压缩至的高度、‘0.4pt‘是图片边框的宽度 (如不需要可调节至0pt)、’picture‘ 是图片文件的名字;可选项、如不需要可删除本行
%\quote{引言(可选项)}                           % 可选项、如不需要可删除本行

% 显示索引号;仅用于在简历中使用了引言
%\makeatletter
%\renewcommand*{\bibliographyitemlabel}{\@biblabel{\arabic{enumiv}}}
%\makeatother

% 分类索引
%\usepackage{multibib}
%\newcites{book,misc}{{Books},{Others}}
%----------------------------------------------------------------------------------
%            内容
%----------------------------------------------------------------------------------
\begin{document}
\begin{CJK}{UTF8}{gbsn}                       % 详情参阅CJK文件包
\maketitle

\section{教育背景}
\cventry{2009年 --\\现在}{硕博连读研究生}{中国科学院上海生命科学研究院计算生物学研究所}{上海}{}{生物中的模式运算研究小组}  % 第3到第6编码可留白
\cventry{2005年 -- 2009年}{生物技术理学学士}{南京大学生命科学学院}{南京}{}{生理学专业方向}


\section{研究经历}
\cventry{2011年6月 -- 7月}{访问学生}{波鸿鲁尔大学}{德国波鸿}{}{生物物理学系生物信息学研究小组}
\cvitem{博士研究\\项目}{%
\begin{itemize}%
  \item Image registration between FTIR image and HE stained image of human colorectal cancer
  \item \href{http://anno.bph.rub.de/}{\color{cyan}Webapp}: a web application for FTIR tissue image annotation
  \item Classification of FTIR tissue images using Tree Classifiers
\end{itemize}}
\cvitem{轮转项目}{Inferring genetic structure of admixed human populations using SNP data}
\cvitem{学士论文}{\href{http://www.picb.ac.cn/wsmotif/}{\color{cyan}WSmotif}: Prediction of human transcription factor binding sites (TFBSs)}
\cvitem{竞赛}{%
\begin{itemize}%
  \item[$\dagger$] 2012 年“有道难题”网易创新大赛,\emph{Candy} 队,作品“{\href{https://github.com/xiaoyong/yirisanxing}{\color{cyan}一日三省}}”,东部赛区三等奖
  \item[$\dagger$] RubyVSPython Planet Conquer 2012 April Contest,冠军
  \item[$\dagger$] Morgan Stanley Code Storm 2011,\emph{Blue Moon} 队,排名 11/20(上海交大赛区)
  \item[$\dagger$] 红帽 2010 大学生 Linux 技能大赛,积极参与奖
\end{itemize}}
\cvitem{开源软件\\项目}{%
\begin{itemize}%
  \item {\href{http://voodoo.xiaoyong.org/}{\color{cyan}Voodoo}: a PICB file search engine based on mlocate}
  \item 查看更多:\url{https://github.com/xiaoyong}
\end{itemize}}


\section{语言}
\cvitemwithcomment{英语}{大学英语六级}{熟悉并适应英语工作环境}
\cvitemwithcomment{汉语}{母语}{普通话以及另外两种方言}


\section{职业技能}
\subsection{学术}
\cvitem{简介}{研究方向是生物医学图像处理,熟悉数字图象处理、模式识别、统计学等知识和技术。}
%\newpage
\subsection{编程}
\cvdoubleitem{Matlab}{精通,使用频繁}{C/C++}{熟练}
\cvdoubleitem{Ruby}{精通,用于简单的脚本和乐趣}{Shell (Bash)}{熟练,用于和计算机交互}
\cvitem{网页开发}{熟悉 HTML, CSS, JavaScript 和 jQuery;了解基于 Ruby 的框架(Ruby on Rails 和 Sinatra)}
\subsection{计算机}
\cvdoubleitem{操作系统}{精通 Linux, Mac OS X 和 Windows 的使用}{办公和排版}{熟练使用微软 Office 和 {\LaTeX}}
%\cvdoubleitem{Text Editing}{Expert on Vim}{}{}


\section{兴趣爱好}
\renewcommand{\listitemsymbol}{-~}            % change the symbol for lists
\cvlistdoubleitem{篮球}{徒步旅行}
\cvlistdoubleitem{电影}{音乐}

%\section{其他 1}
%\cvlistitem{项目 1}
%\cvlistitem{项目 2}
%\cvlistitem{项目 3}

%\renewcommand{\listitemsymbol}{-}             % 改变列表符号

%\section{其他 2}
%\cvlistdoubleitem{项目 1}{项目 4}
%\cvlistdoubleitem{项目 2}{项目 5\cite{book1}}
%\cvlistdoubleitem{项目 3}{} 
% 来自BibTeX文件但不使用multibib包的出版物
%\renewcommand*{\bibliographyitemlabel}{\@biblabel{\arabic{enumiv}}}% BibTeX的数字标签
\nocite{*}
\bibliographystyle{plain}
\bibliography{publications}                    % 'publications' 是BibTeX文件的文件名

% 来自BibTeX文件并使用multibib包的出版物
%\section{出版物}
%\nocitebook{book1,book2}
%\bibliographystylebook{plain}
%\bibliographybook{publications}               % 'publications' 是BibTeX文件的文件名
%\nocitemisc{misc1,misc2,misc3}
%\bibliographystylemisc{plain}
%\bibliographymisc{publications}               % 'publications' 是BibTeX文件的文件名

%\clearpage
\end{CJK}
\end{document}


%% 文件结尾 `template-zh.tex'.
